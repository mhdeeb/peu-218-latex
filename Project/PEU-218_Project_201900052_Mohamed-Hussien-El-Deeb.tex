\documentclass[12pt]{article}
\usepackage[svgnames,x11names,table]{xcolor}
\usepackage{hyperref}
\usepackage{graphicx}
\usepackage{parskip}
\usepackage{float}
\usepackage{amsmath}
\usepackage{amssymb}
\usepackage{enumitem}
\usepackage[thicklines]{cancel}

\hypersetup{
    colorlinks,
    citecolor=blue,
    filecolor=black,
    linkcolor=black,
    urlcolor=RoyalBlue4,
}

\title{Creating a Simple Vector Calculus Framework to Verify Stokes' Curl Theorem}
\author{Mohamed Hussien El-Deeb (201900052)}
\date{\today}

\begin{document}

\maketitle
\tableofcontents
\hypersetup{linkcolor=RoyalBlue4}

\newpage
\section{Introduction}

Stokes' curl theorem, also known as the Kelvin-Stokes theorem after \href{https://en.wikipedia.org/wiki/Lord_Kelvin}{Lord Kelvin} and
\href{https://en.wikipedia.org/wiki/Sir_George_Stokes,_1st_Baronet}{George Stokes}, is a theorem in vector calculus on \({\displaystyle \mathbb {R} ^{3}}\). Given
a vector field, the theorem relates the integral of the curl of the vector field over some
surface, to the line integral of the vector field around the boundary of the surface. This theorem can be
stated as: The line integral of a vector field over a loop is equal to the surface
integral of its curl over the enclosed surface. This unlocks integration shortcuts that greatly simplifies other theorems and speeds up computations.
This project is an attempt to build a simple vector calculus framework to verify and visualize
Stokes' curl theorem using basic numerical methods and the web graphics library \href{https://threejs.org/}{three.js}.

\newpage
\section{Stokes' Curl Theorem}
\subsection{Definition}

Stokes' theorem states that the line integral of a vector field \(\mathbf{F}\) around a closed curve \(C\) is equal to the surface integral of the curl of \(\mathbf{F}\) over any surface \(S\) bounded by \(C\). Mathematically, the theorem can be stated as:

\[
    \oint_{C} \mathbf{F} \cdot d\mathbf{r} = \iint_{S} (\nabla \times \mathbf{F}) \cdot d\mathbf{S}
\]

where:

\begin{itemize}
    \item \(\mathbf{F}\) is a differentiable vector field.
    \item \(C\) is a piecewise-smooth, positively oriented simple closed curve in space.
    \item \(S\) is a piecewise-smooth, positively oriented surface in space bounded by \(C\).
    \item \(d\mathbf{r}\) is the differential arc vector along \(C\).
    \item \(d\mathbf{S}\) is the differential area vector of \(S\).
\end{itemize}

\newpage
\section{Integration over a Path}

Since computers are discrete, we need to approximate the integral of a function over a path using numerical methods which means we will revert to using summation instead of integration while noting the values converge as the step size decreases.

Suppose we want to sum up the values of a function \(f(x, y, z)\) along a path \(C\) in space.
We can approximate the path by dividing it into many small line segments and approximate the function by a constant value over each segment.
The sum of the values of the function over all segments is the integral of the function over the path.
Numerically, the integral can be approximated by the following formula:

\[
    \sum_{i=0}^{n} f(L_i)
\]

where:

\begin{itemize}
    \item \(f\) is the function to be integrated.
    \item \(L_i\) is a small line segment on the path.
    \item \(n\) is the integer number of line segments.
\end{itemize}

\(L_i\) is defined as follows:

\[
    L_i = \left\langle C\left(i \Delta t\right), C\left((i + 1)\Delta t\right)\right\rangle
\]

where:

\begin{itemize}
    \item \(C(t)\) is the parametric equation of the path.
    \item \(\Delta t\) is the step size.
\end{itemize}

We can choose our parameter \(t\) to be \(0 \leq t \leq 1\) meaning we can substitute \(\Delta t\) with
\(\frac{1}{n}\).

Note that it is possible to convert parameter limits from \([a, b]\) to \([0, 1]\) using the following formula as long as \(a \neq b\):

\[
    P(t), \quad a \leq t \leq b \quad \rightarrow \quad P\left(\frac{t - a}{b - a}\right), \quad 0 \leq t \leq 1
\]

Now we arrive at the following formula:

\[
    \sum_{i=0}^{n} f\left(L_i\right)
\]

\[
    L_i = \left\langle C\left(\frac{i}{n}\right), C\left(\frac{i + 1}{n}\right)\right\rangle
\]

The higher the value of \(n\), the more accurate the approximation.

\newpage
\section{Integration over a Surface}

Similarly, we can approximate the integral of a function \(f(x, y, z)\) over a surface \(S\) using numerical methods.
We can divide the surface into small triangles and approximate the function by a constant value over each triangle.
The sum of the values of the function over all triangles converges to the integral of the function over the surface
as the limit of the step size reach 0.

Unlike the path, the surface is a 2D object in 3D space which means we need two parameters to define a point on the surface and
the triangle construction will be a bit more complex.

Numerically, the integral can be approximated by the following formula:

\[
    \sum_{i, j=0}^{n, m} f(T_{ij})
\]

where:

\begin{itemize}
    \item \(f\) is the function to be integrated.
    \item \(T_i\) is a small triangle on the surface.
    \item \(n\) is the integer number of triangles.
\end{itemize}

\(T_i\) is defined as follows:

\[
    T_i = \left\langle S\left(i \Delta u, j \Delta v\right), S\left((i + 1)\Delta u, j \Delta v\right), S\left(i \Delta u, (j + 1) \Delta v\right)\right\rangle
\]

\newpage
\section{Scaler Field}
\subsection{Definition}



\subsection{Gradient of a Scalar Field}



\newpage
\section{Vector Field}
\subsection{Definition}



\subsection{Divergence of a Vector Field}



\subsection{Curl of a Vector Field}



\newpage
\section{Result}
\subsection{Error Analysis}



\subsection{Discussion}



\newpage
\section{Conclusion}



\newpage
\section{References}
\bibliographystyle{plain}
\bibliography{references}
\nocite{El-Deeb_PEU-218_Assignments}

\end{document}