\documentclass[12pt]{article}
\usepackage[svgnames,x11names,table]{xcolor}
\usepackage{hyperref}
\usepackage{graphicx}
\usepackage{parskip}
\usepackage{float}
\usepackage{amsmath}
\usepackage{amssymb}
\usepackage{enumitem}
\usepackage[thicklines]{cancel}

\hypersetup{
    colorlinks,
    citecolor=blue,
    filecolor=black,
    linkcolor=black,
    urlcolor=RoyalBlue4,
}

\title{PEU 218 Assignment 4}
\author{Mohamed Hussien El-Deeb (201900052)}
\date{\today}

\DeclareMathOperator{\sech}{sech}
\DeclareMathOperator{\csch}{csch}

\begin{document}

\maketitle
\tableofcontents
\hypersetup{linkcolor=RoyalBlue4}

\newpage
\section{Question 1}

\subsection{Problem}

Paraboloidal coordinates \(u, v, \varphi \) are defined in terms of Cartesian coordinates

\[
    \begin{aligned}
         & x=\alpha \beta \cos \varphi, \quad y=\alpha \beta \sin \varphi, \quad z=\frac{1}{2}\left(\alpha^2-\beta^2\right) \\
         & \text { where } 0 \leq \alpha \leq \infty, \quad 0 \leq \beta \leq \infty, \quad 0 \leq \varphi \leq 2 \pi
    \end{aligned}
\]

Prove that the \(\alpha \) -component of \(\vec{\nabla} \times \vec{A}\) is

\[
    \frac{1}{{\left(\alpha^2+\beta^2\right)}^{1 / 2}}\left(\frac{A_{\varphi}}{\beta}+\frac{\partial \varphi}{\partial \beta}\right)-\frac{1}{\alpha \beta} \frac{\partial A_\beta}{\partial \varphi}
\]

\subsection{Solution}

\[
    \vec{\nabla} \times \vec{A} = \frac{1}{h_1 h_2 h_3}\begin{vmatrix}
        h_1 \hat{e}_1                 & h_2 \hat{e}_2                 & h_3 \hat{e}_3                 \\
        \frac{\partial}{\partial q_1} & \frac{\partial}{\partial q_2} & \frac{\partial}{\partial q_3} \\
        A_1 h_1                       & A_2 h_2                       & A_3 h_3
    \end{vmatrix}
\]

\[
    {(\vec{\nabla} \times \vec{A})}_\alpha =
    \frac{1}{h_\beta h_\varphi}\left(\frac{\partial}{\partial \beta}\left(A_\varphi h_\varphi\right) - \frac{\partial}{\partial \varphi}\left(A_\beta h_\beta\right)\right)
\]

\[
    = \frac{1}{h_\beta h_\varphi}\left(\frac{\partial A_\varphi}{\partial \beta} h_\varphi + \frac{\partial h_\varphi}{\partial \beta} A_\varphi - \frac{\partial A_\beta}{\partial \varphi} h_\beta - \frac{\partial h_\beta}{\partial \varphi} A_\beta\right)
\]

\[
    h_\beta = \left\lvert \frac{\partial \vec{r}}{\partial \beta}\right\rvert
    = \sqrt{
    {\left(\frac{\partial x}{\partial \beta}\right)}^2
    + {\left(\frac{\partial y}{\partial \beta}\right)}^2
    + {\left(\frac{\partial z}{\partial \beta}\right)}^2
    }
\]

\[
    = \sqrt{
    {\left(\alpha \cos \varphi\right)}^2
    + {\left(\alpha \sin \varphi\right)}^2
    + {\left(\beta\right)}^2
    } = \sqrt{\alpha^2 + \beta^2}
\]

\[
    h_\varphi = \left\lvert \frac{\partial \vec{r}}{\partial \varphi}\right\rvert
    = \sqrt{
    {\left(\frac{\partial x}{\partial \varphi}\right)}^2
    + {\left(\frac{\partial y}{\partial \varphi}\right)}^2
    + {\left(\frac{\partial z}{\partial \varphi}\right)}^2
    }
\]

\[
    = \sqrt{
    {\left(\alpha \beta \sin \varphi\right)}^2
    + {\left(\alpha \beta \cos \varphi\right)}^2
    } = \alpha \beta
\]

\[
    \frac{\partial h_\beta}{\partial \varphi}
    = \frac{\partial}{\partial \varphi}\left(\sqrt{\alpha^2 + \beta^2}\right) = 0
\]

\[
    \frac{\partial h_\varphi}{\partial \beta}
    = \frac{\partial}{\partial \beta}\left(\alpha \beta\right) = \alpha
\]

\[
    {(\vec{\nabla} \times \vec{A})}_\alpha =
    \frac{1}{\alpha \beta \sqrt{\alpha^2 + \beta^2}}
    \left(
    \frac{\partial A_\varphi}{\partial \beta} \alpha \beta
    + \alpha A_\varphi
    - \frac{\partial A_\beta}{\partial \varphi} \sqrt{\alpha^2 + \beta^2}
    \right)
\]

\[
    = \frac{1}{{\left(\alpha^2+\beta^2\right)}^{1 / 2}}
    \left(
    \frac{A_{\varphi}}{\beta}
    + \frac{\partial A_\varphi}{\partial \beta}
    \right)-\frac{1}{\alpha \beta} \frac{\partial A_\beta}{\partial \varphi}
\]

\newpage
\section{Question 2}

\subsection{Problem}

Using spherical coordinates, evaluate

\[
    \nabla^2\left(\vec{\nabla} \cdot \frac{\vec{r}}{r^2}\right)
\]



\subsection{Solution}

\[
    \vec{r} = \left\langle r, 0, 0 \right\rangle
\]

\[
    \frac{\vec{r}}{r^2} =
    \vec{r} = \left\langle \frac{1}{r}, 0, 0 \right\rangle
\]

\[
    \vec{\nabla} \cdot \vec{u}=
    \frac{1}{r^2} \frac{\partial\left(r^2 u_r\right)}{\partial r}+\frac{1}{r \sin \theta} \frac{\partial\left(\sin \theta u_\theta\right)}{\partial \theta}+\frac{1}{r \sin \theta} \frac{\partial\left(u_{\varphi}\right)}{\partial \varphi}
\]

\[
    \vec{\nabla} \cdot \frac{\vec{r}}{r^2} =
    \frac{1}{r^2} \frac{\partial r}{\partial r} = \frac{1}{r^2}
\]

\[
    \nabla^2 f=\frac{1}{r^2} \frac{\partial}{\partial r}\left(r^2 \frac{\partial f}{\partial r}\right)+\frac{1}{r^2 \sin \theta} \frac{\partial}{\partial \theta}\left(\sin \theta \frac{\partial f}{\partial \theta}\right)+\frac{1}{r^2 \sin ^2 \theta} \frac{\partial^2 f}{\partial \varphi^2}
\]

\[
    \nabla^2 \frac{1}{r^2} =
    \frac{1}{r^2} \frac{\partial}{\partial r}\left(r^2 \frac{\partial \frac{1}{r^2}}{\partial r}\right) =
    - \frac{2}{r^2} \frac{\partial}{\partial r}\left(r^{-1}\right) = \frac{2}{r^4}
\]

\[
    \nabla^2\left(\vec{\nabla} \cdot \frac{\vec{r}}{r^2}\right) = \frac{2}{r^4}
\]

\newpage
\section{Question 3}

\subsection{Problem}

We introduce the so-called spheroidal coordinates \((\eta, \theta, \varphi)\) by the following equations expressed in rectangular coordinates

\[
    \begin{gathered}
        x=a \sinh \eta \sin \theta \cos \varphi, \\
        y=a \sinh \eta \sin \theta \sin \varphi, \\
        z=a \cosh \eta \cos \theta
    \end{gathered}
\]

where \(0 \leq \eta \leq \infty, \quad 0 \leq \theta \leq \pi, \quad 0 \leq \varphi \leq 2 \pi \)

a) Show that this coordinate system is orthogonal.

b) Find the scale factors.

c) Show that the function \(f(\eta, \theta, \varphi)=\ln \tanh \left(\frac{\eta}{2}\right)\) is a solution of Laplace's equation.

\subsection{Solution}

a)

\[
    \vec{r} = a \left\langle \sinh \eta \sin \theta \cos \varphi, \sinh \eta \sin \theta \sin \varphi, \cosh \eta \cos \theta \right\rangle
\]

\[
    \vec{e}_\eta = \frac{\partial \vec{r}}{\partial \eta} = a \left\langle \cosh \eta \sin \theta \cos \varphi, \cosh \eta \sin \theta \sin \varphi, \sinh \eta \cos \theta \right\rangle
\]

\[
    \vec{e}_\theta = \frac{\partial \vec{r}}{\partial \theta} = a \left\langle \sinh \eta \cos \theta \cos \varphi, \sinh \eta \cos \theta \sin \varphi, -\cosh \eta \sin \theta \right\rangle
\]

\[
    \vec{e}_\varphi = \frac{\partial \vec{r}}{\partial \varphi} = a \left\langle -\sinh \eta \sin \theta \sin \varphi, \sinh \eta \sin \theta \cos \varphi, 0 \right\rangle
\]

\[
    \frac{\partial \vec{r}}{\partial \eta} \cdot \frac{\partial \vec{r}}{\partial \theta} =
    a^2 \left(\sinh \eta \cosh \eta \sin \theta \cos \theta - \sinh \eta \cosh \eta \sin \theta \cos \theta\right) = 0
\]

\[
    \frac{\partial \vec{r}}{\partial \eta} \cdot \frac{\partial \vec{r}}{\partial \varphi} =
\]

\[
    a^2 \left(-\sinh \eta \cosh \eta \sin \varphi \cos \varphi \sin^2 \theta + \sinh \eta \cosh \eta \sin \varphi \cos \varphi \sin^2 \theta\right) = 0
\]

\[
    \frac{\partial \vec{r}}{\partial \theta} \cdot \frac{\partial \vec{r}}{\partial \varphi} =
\]

\[
    a^2 \left(-\sinh^2 \eta \sin \theta \cos \theta \sin \varphi \cos \varphi + \sinh^2 \eta \sin \theta \cos \theta \sin \varphi \cos \varphi\right) = 0
\]

Since the dot product of different basis vectors is zero, the coordinate system is orthogonal.

b)

\[
    \cosh^2 \eta = 1 + \sinh^2 \eta
\]

\[
    h_\eta = \left\lvert \frac{\partial \vec{r}}{\partial \eta}\right\rvert
    = a \sqrt{\cosh^2 \eta \sin^2 \theta + \sinh^2 \eta \cos^2 \theta}
\]

\[
    = a \sqrt{(1 + \sinh^2 \eta) \sin^2 \theta + \sinh^2 \eta \cos^2 \theta}
    = a \sqrt{\sinh^2 \eta + \sin^2 \theta}
\]

\[
    h_\theta = \left\lvert \frac{\partial \vec{r}}{\partial \theta}\right\rvert
    = a \sqrt{\sinh^2 \eta \cos^2 \theta + \cosh^2 \eta \sin^2 \theta} = h_\eta
\]

\[
    h_\varphi = \left\lvert \frac{\partial \vec{r}}{\partial \varphi}\right\rvert
    = a \sinh \eta \sin \theta
\]

c)

\[
    \nabla^2 f = \frac{1}{h_1 h_2 h_3}\left[\frac{\partial}{\partial q_1}\left(\frac{h_2 h_3}{h_1} \frac{\partial f}{\partial q_1}\right)+\frac{\partial}{\partial q_2}\left(\frac{h_3 h_1}{h_2} \frac{\partial f}{\partial q_2}\right)+\frac{\partial}{\partial q_3}\left(\frac{h_1 h_2}{h_3} \frac{\partial f}{\partial q_3}\right)\right]
\]

In this coordinate system, \(h_1 = h_2 = h_\eta, h_3 = h_\varphi \),

\[
    \nabla^2 f = \frac{1}{h_\eta^2 h_\varphi}\left[\frac{\partial}{\partial \eta}\left(h_\varphi \frac{\partial f}{\partial \eta}\right)+\frac{\partial}{\partial \theta}\left(h_\varphi \frac{\partial f}{\partial \theta}\right)+\frac{\partial}{\partial \varphi}\left(\frac{h_\eta^2}{h_\varphi} \frac{\partial f}{\partial \varphi}\right)\right]
\]

Since we are solving laplace equation, we need only to focus on what is inside the square brackets.

\[
    \frac{\partial}{\partial \eta}\left(h_\varphi \frac{\partial f}{\partial \eta}\right)
    + \frac{\partial}{\partial \theta}\left(h_\varphi \frac{\partial f}{\partial \theta}\right)
    + \frac{\partial}{\partial \varphi}\left(\frac{h_\eta^2}{h_\varphi} \frac{\partial f}{\partial \varphi}\right)
\]

Since f is a function of \(\eta \) only, we can take our solution to be a function of only \(\eta \) so second and third terms will be zero.

\[
    \frac{\partial}{\partial \eta}\left(h_\varphi \frac{\partial f}{\partial \eta}\right)
\]

We could also induce that,

\[
    h_\varphi \frac{\partial f}{\partial \eta} = g(\theta, \varphi)
\]

\[
    \frac{\partial f}{\partial \eta}
    = \frac{\partial}{\partial \eta}\left(\ln \tanh \left(\frac{\eta}{2}\right)\right)
    = \frac{1}{2} \sech^2 \frac{\eta}{2} \coth \left(\frac{\eta}{2}\right)
    = \frac{1}{2} \csch \frac{\eta}{2} \sech \frac{\eta}{2}
    = \csch \eta
\]

\[
    h_\varphi \frac{\partial f}{\partial \eta}
    = a \sin \theta \sinh \eta \csch \eta
    = a \sin \theta
\]

Since g is not a function of \(\eta \), we could deduce that f is a solution of Laplace's equation.

\newpage
\bibliographystyle{plain}
\bibliography{references}
\nocite{El-Deeb_PEU-218_Assignments}

\end{document}