\documentclass[12pt]{article}
\usepackage[svgnames,x11names,table]{xcolor}
\usepackage{hyperref}
\usepackage{graphicx}
\usepackage{parskip}
\usepackage{float}
\usepackage{amsmath}
\usepackage{amssymb}
\usepackage{enumitem}
\usepackage[thicklines]{cancel}

\hypersetup{
    colorlinks,
    citecolor=blue,
    filecolor=black,
    linkcolor=black,
    urlcolor=RoyalBlue4,
}

\title{PEU 218 Assignment 2}
\author{Mohamed Hussien El-Deeb (201900052)}
\date{\today}

\begin{document}

\maketitle
\tableofcontents
\hypersetup{linkcolor=RoyalBlue4}

\newpage
\section{Question 1}

\subsection{Problem}

Evaluate
\[
    \int_C \left(2y x^2 - 4x\right) d r
\]

Where \(C\) is lower half of the circle centered at the origin of radius 3 with clockwise
rotation.

\subsection{Solution}



\newpage
\section{Question 2}

\subsection{Problem}

Evaluate \(\int_C \vec{F} \cdot d \vec{r}\), where
\(\vec{F}=\left\langle y, 3y^3 - x, z\right\rangle \) and the path \(C\) is defined by
\(C(t) = \left\langle t, t^n, 0\right\rangle, 0 \leq t \leq 1\) where \(n = 1, 2, 3, .. \)

\subsection{Solution}



\newpage
\section{Question 3}

\subsection{Problem}

Evaluate \(\int_C \vec{F} \cdot d \vec{r}\), where \(\vec{F} = \langle xy, 1 + 3y, 0\rangle \) and
\(C\) is the line segment from \((0, -4)\) to \((-2, -4)\) followed by portion of \(y = -x^2\)
from \(x = -2\) to \(x = 2\) which is in turn followed by the line segment from \((2, -4)\)
to \((5, 1)\).

\subsection{Solution}



\newpage
\section{Question 4}

\subsection{Problem}

Evaluate \(\iint_S 2y\ dS\) where \(S\) is the portion \(y^2 + z^2 = 4\) between \(x = 0\)
and \(x = 3 - z\).

\subsection{Solution}



\newpage
\section{Question 5}

\subsection{Problem}

Let the temperature of a point in space be given by \(T(x, y, z)= 3x^2 + 3z^2\).
Compute the heat flux across the surface \(x^2 + z^2 = 2, 0 \leq y \leq 2\), if \(k = 1\).
(Give a physical explanation to justify the sign of your result?)

\subsection{Solution}



\newpage
\section{Question 6}

\subsection{Problem}

Evaluate \(\iint_S \vec{F} \cdot d S\) where
\(\vec{F} = y \hat{\imath} + 2x \hat{\jmath} + (z - 8) \hat{k}\) and \(S\) is the surface of the
solid bounded by \(4x + 2y + z = 8, z = 0, y = 0\) and \(x = 0\) with the positive orientation.
Note that all four surfaces of the solid are included in \(S\).

\subsection{Solution}



\newpage
\bibliographystyle{plain}
\bibliography{references}
\nocite{El-Deeb_PEU-218_Assignments}

\end{document}